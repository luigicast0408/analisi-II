\documentclass{article}
\usepackage{amsmath}

\begin{document}

\section*{Esempio: Determinazione dei Punti Stazionari}

Consideriamo la funzione:
\[
f(x, y) = x^2 + y^2 - 4x - 6y
\]

\subsection*{1. Derivate Prime}

Calcoliamo le derivate parziali:

1. **Derivata parziale rispetto a \( x \)**:
   \[
   f_x = \frac{\partial f}{\partial x} = 2x - 4
   \]

2. **Derivata parziale rispetto a \( y \)**:
   \[
   f_y = \frac{\partial f}{\partial y} = 2y - 6
   \]

\subsection*{2. Derivate Seconde}

Calcoliamo le derivate seconde:

1. **Derivata seconda rispetto a \( x \)**:
   \[
   f_{xx} = \frac{\partial^2 f}{\partial x^2} = 2
   \]

2. **Derivata seconda rispetto a \( y \)**:
   \[
   f_{yy} = \frac{\partial^2 f}{\partial y^2} = 2
   \]

3. **Derivata mista** (rispetto a \( x \) e \( y \)):
   \[
   f_{xy} = \frac{\partial^2 f}{\partial y \partial x} = 0
   \]

4. **Derivata mista** (rispetto a \( y \) e \( x \)):
   \[
   f_{yx} = \frac{\partial^2 f}{\partial x \partial y} = 0
   \]

\subsection*{3. Matrice Hessiana}

La matrice Hessiana \( H \) è definita come segue:

\[
H = \begin{bmatrix}
f_{xx} & f_{xy} \\
f_{yx} & f_{yy}
\end{bmatrix}
= \begin{bmatrix}
2 & 0 \\
0 & 2
\end{bmatrix}
\]

\subsection*{4. Determinante della Matrice Hessiana}

Il determinante della matrice Hessiana è dato da:

\[
D = \det(H) = f_{xx} f_{yy} - f_{xy} f_{yx} = (2)(2) - (0)(0) = 4
\]

\subsection*{5. Analisi del Punto Stazionario}

Abbiamo trovato il punto stazionario \( (2, 3) \). Per analizzare la natura del punto stazionario, utilizziamo i seguenti criteri:

\begin{itemize}
    \item Se \( D > 0 \) e \( f_{xx} > 0 \), allora il punto è un minimo locale.
    \item Se \( D > 0 \) e \( f_{xx} < 0 \), allora il punto è un massimo locale.
    \item Se \( D < 0 \), allora il punto è un punto di sella.
    \item Se \( D = 0 \), la test non è conclusivo.
\end{itemize}

\textbf{Risultato:} Nel nostro caso:
\begin{align*}
D & = 4 > 0 \\
f_{xx} & = 2 > 0
\end{align*}

Quindi, il punto \( (2, 3) \) è un \textbf{minimo locale} della funzione \( f(x, y) = x^2 + y^2 - 4x - 6y \).

\end{document}
