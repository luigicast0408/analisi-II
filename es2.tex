\documentclass{article}
\usepackage{amsmath}

\begin{document}

\section*{Derivate seconde parziali}

Le derivate seconde parziali si riferiscono alla derivazione di funzioni di più variabili. Ecco come calcolarle:

\subsection*{1. Derivate seconde parziali pure}

Le derivate seconde parziali pure si ottengono derivando due volte la funzione rispetto alla stessa variabile.

Se \( f(x, y) \) è una funzione di due variabili, le derivate seconde pure sono:

- \textbf{Rispetto a \( x \)}:
\[
f_{xx} = \frac{\partial^2 f}{\partial x^2}
\]

- \textbf{Rispetto a \( y \)}:
\[
f_{yy} = \frac{\partial^2 f}{\partial y^2}
\]

\subsection*{2. Derivate seconde parziali miste}

Le derivate seconde parziali miste si ottengono derivando una volta rispetto a una variabile e poi una volta rispetto a un'altra variabile. Le derivate seconde miste di una funzione di due variabili sono:

- \textbf{Primo rispetto a \( x \) e poi rispetto a \( y \)}:
\[
f_{xy} = \frac{\partial^2 f}{\partial y \partial x}
\]

- \textbf{Primo rispetto a \( y \) e poi rispetto a \( x \)}:
\[
f_{yx} = \frac{\partial^2 f}{\partial x \partial y}
\]

\subsection*{Proprietà}

Una delle proprietà fondamentali delle derivate seconde parziali è il \textbf{teorema di Schwarz}, che afferma che, sotto certe condizioni di continuità, le derivate seconde miste sono uguali:
\[
f_{xy} = f_{yx}
\]

\subsection*{Esempio}

Consideriamo la funzione:
\[
f(x, y) = x^2y + 3xy^2
\]

\textbf{Calcolo delle derivate seconde pure}:

1. \textbf{Derivata prima rispetto a \( x \)}:
   \[
   f_x = \frac{\partial f}{\partial x} = 2xy + 3y^2
   \]

2. \textbf{Derivata seconda rispetto a \( x \)}:
   \[
   f_{xx} = \frac{\partial^2 f}{\partial x^2} = 2y
   \]

3. \textbf{Derivata prima rispetto a \( y \)}:
   \[
   f_y = \frac{\partial f}{\partial y} = x^2 + 6xy
   \]

4. \textbf{Derivata seconda rispetto a \( y \)}:
   \[
   f_{yy} = \frac{\partial^2 f}{\partial y^2} = 6x
   \]

\textbf{Calcolo delle derivate seconde miste}:

1. \textbf{Derivata prima rispetto a \( x \), poi \( y \)}:
   \[
   f_{xy} = \frac{\partial}{\partial y}(f_x) = \frac{\partial}{\partial y}(2xy + 3y^2) = 2x + 6y
   \]

2. \textbf{Derivata prima rispetto a \( y \), poi \( x \)}:
   \[
   f_{yx} = \frac{\partial}{\partial x}(f_y) = \frac{\partial}{\partial x}(x^2 + 6xy) = 2x + 6y
   \]

In questo caso, vediamo che \( f_{xy} = f_{yx} \).

\end{document}
