\documentclass{article}
\usepackage{amsmath}
\usepackage{amssymb} % Per simboli come \mathbb{N}

\usepackage{amsthm}
\usepackage{amsmath} % Per formattazioni matematiche avanzate, se non già incluso.


\begin{document}

\section*{Serie}

\subsection*{Serie numerica}
\[
a_1 + a_2 + a_3 + \dots + a_n, \quad \sum_{n=1}^{\infty} a_n
\]

\subsection*{Significato di $s_n$}
\begin{align}
    s_1 &= a_1 \\
    s_2 &= a_1 + a_2 \\
    s_n &= a_1 + a_2 + \dots + a_n, \quad n \in \mathbb{N}, \quad n > 1 \\
    \text{Somma parziale } s_n &= \sum_{i=1}^n a_i
\end{align}

- Se $s_n \to s$, la serie converge ed ha somma $s$:
\[
\sum_{n=1}^{\infty} a_n = s
\]

\begin{itemize}
    \item Se $s_n \to +\infty$ (o $-\infty$), la serie diverge positivamente (o negativamente).
    \item Se $\{s_n\}$ non è regolare, la serie è indeterminata, oscillante o non regolare.
\end{itemize}

Studiare il carattere della serie numerica significa stabilire se è regolare o meno, e, in caso di regolarità, se converge o diverge.

Consideriamo la serie:
\[
\sum_{n=1}^{+\infty} (-1)^n = \{ (-1)^1 + (-1)^2 + (-1)^3 + \dots \}
\]
Questa serie è indeterminata, poiché la somma parziale ennesima è data da:
\[
s_n = 
\begin{cases}  
0, & \text{se } n \text{ è pari} \\ 
-1, & \text{se } n \text{ è dispari}
\end{cases}
\]
Dunque, $\{s_n\}$ non è dotata di limite.

\subsection*{Serie telescopica}
Sia $\{s_n\}$ una successione numerica. La serie telescopica è definita come:
\[
\sum_{n=1}^{+\infty} (x_n - x_{n+1})
\]
La sua somma parziale di posto $n$ è 
\[
s_n = (x_1 - x_2) + (x_2 - x_3) + \dots + (x_{n-1} - x_n) = x_1 - x_n
\]
La serie quindi:
\begin{itemize} 
    \item converge ed ha somma $x_1 - l$ se $\lim x_n = l \in \mathbb{R}$
    \item diverge positivamente (negativamente) se $\lim x_n = +\infty$ (o $-\infty$)
    \item è indeterminata, se non esiste $\lim x_n$
\end{itemize}

\subsection*{Serie di Mengoli}
Posto $x_n = \frac{1}{n} \quad \forall\ n \in \mathbb{N}$
\[
\sum_{n=1}^{\infty} \left( \frac{1}{n} - \frac{1}{n+1} \right) = \sum_{n=1}^{+\infty} \frac{1}{n\left( n+1\right)}
\]
\[
\textbf{converge a 1}
\]

\section*{Serie geometrica di ragione $q$}
\begin{align*}
    &\text{Sia } q \in \mathbb{R}. \\ 
    &\text{La serie } \sum_{n=1}^{+\infty} q^{n-1} \text{ rappresenta la somma:} \\
    &1 + q + q^2 + \cdots + q^{n-1} + \cdots \\
    &\text{Ad esempio, posto } q = \frac{1}{2}, \text{ otteniamo:} \\
    &1 + \frac{1}{2} + \frac{1}{4} + \frac{1}{8} + \cdots \\
    &\text{Consideriamo ora la somma parziale dei primi } n \text{ termini:} \\
    &S_n = 1 + q + q^2 + \cdots + q^{n-1} \\
    &\text{Moltiplichiamo } S_n \text{ per } q: \\
    &q S_n = q + q^2 + \cdots + q^{n-1} + q^n \\
    &\text{Sottraendo membro a membro si ottiene:} \\
    &S_n - q S_n = (1 + q + q^2 + \cdots + q^{n-1}) - (q + q^2 + \cdots + q^n) \\
    &S_n (1 - q) = 1 - q^n \\
    &\text{da cui si ottiene:}\\
    & S_n = \frac{1 - q^n}{1 - q} \\
    &\text{Ad esempio, posto } q = \frac{1}{2}, \text{ otteniamo:} \\
    &S_n = \frac{1 - \frac{1}{2^n}}{1 - \frac{1}{2}} = 2 \left(1 - \frac{1}{2^n}\right) \\
    &\text{Se invece } q = 1, \text{ otteniamo:} \\
    &S_n = 1 + 1 + 1 + \cdots + 1 = n, \quad \text{con } 1 \text{ sommato } n \text{ volte}. \\
    &\text{In generale, quindi:} \\
    & S_n = 
    \begin{cases}
        \frac{1 - q^n}{1 - q} & \text{se } q \neq 1 \\
        n & \text{se } q = 1 \\
    \end{cases}
\end{align*}

\[
    \lim_{n \to \infty} S_n = 
    \begin{cases}
        +\infty & \text{se } q > 1, \quad (\lim_{n \to \infty} q^n = +\infty) \\
        +\infty & q = 1 \\
        \frac{1}{1 - q} & \text{se } |q| < 1, \quad (\lim q^n = 0) \\
        \nexists & \text{se } q \leq -1  \quad (\nexists \lim q^n)\\
    \end{cases}
\]

\begin{align*}
    \textbf{Riassumendo: La serie geometrica di ragione } q \\
    \begin{cases}
        \textbf{Diverge positivamente} & \text{se } q \geq 1 \\
        \textbf{Convergente con somma } \frac{1}{1 - q} & \text{se } |q| < 1 \\
        \textbf{Oscillante} & \text{se } q \leq -1
    \end{cases}
\end{align*}

\subsection*{Esempi di serie geometriche}
\begin{align*}
    \sum_{n=1}^{\infty} \left( \frac{-3}{4}\right )^{n-1}, \quad q = \frac{3}{4} \\
    &= \frac{1}{1-\left(-\frac{3}{4} \right)} = \frac{4}{7}  \text(Convergente con somma )\\
\end{align*}

\begin{align*}
    \sum_{n=1}^\infty \left( -2\right )^{n-1} \text{ è oscillante} \\
    \sum_{n=1}^\infty 3^{n-1} \text{ è divergente positivamente} \\
\end{align*}

\section*{Serie Armonica}
\[
    \sum_{n=1}^{+\infty} \frac{1}{n}
\]
carattere: diverge a $+\infty$

\[
    s_n = 1 + \frac{1}{2} + \frac{1}{3} + \cdots + \frac{1}{n} \quad \forall n \in \mathbb{N}
\]

\section*{Serie Armonica generalizzata}
\[
    \sum_{n=1}^{\infty} \frac{1}{n^x} \quad x \in \mathbb{R}
\]
Carattere:
\[
    \begin{cases}
        \text{converge} & \text{se } x > 1 \\
        \text{diverge} & \text{se } x \leq 1
    \end{cases}
\]

\section*{Operazioni con le serie}
\begin{align*}
    &\sum_{n=1}^{+\infty} a_n  = A \in \mathbb{R}\\
    &\sum_{n=1}^{+\infty} b_n = B \in \mathbb{R} \\
    &\sum_{n=1}^{+\infty} (a_n +b_n) = A+B
\end{align*}

\begin{align*}
    &\sum_{n=1}^{+\infty} a_n  = +\infty (-\infty)\\
    &\sum_{n=1}^{+\infty} b_n = +\infty (-\infty) \\
    &\sum_{n=1}^{+\infty} (a_n +b_n) = +\infty (-\infty)
\end{align*}

\subsection*{Esempi di somma con le serie}
\[
    \sum_{n=1}^{\infty} \left[\left( \frac{-1}{2}\right)^{n-1} + \frac{1}{n}\right] (1)
\]

\begin{align*}
        &\text{Si procede scindendo in due la serie} \\
        &\sum_{n=1}^{\infty} \left(-\frac{1}{2} \right)^{n-1}  &\textbf{serie geometrica di ragione $\left( -\frac{1}{2}\right)$ convergente}\\  
        &\sum_{n=1}^{\infty} \frac{1}{n} &\textbf{Serie amonica diverge positivamente} 
\end{align*}
    
\begin{enumerate}
        \item \textbf{serie geometrica di ragione $\left( -\frac{1}{2}\right)$ convergente} (-)
        \item \textbf{Serie amonica diverge positivamente} (+)
\end{enumerate}
    $\textbf{da questi risultati si ha che la serie diverge }(-)$

\begin{align*}
        &\sum_{n=1}^{\infty} \left[\left( -\frac{1}{2}\right)^{n-1} + \left (\frac{1}{3}\right )^{n-1} \right] (2) \\
        &\text{Si ha una somma di due serie geometriche} \\
\end{align*}

    
    
\begin{enumerate}
        \item Serie geometrica di ragione $\left(-\frac{1}{2}\right)^{n-1}$
        \item Serie geometrica di ragione $\left(\frac{1}{3}\right)^{n-1}$
\end{enumerate}
    \[
        \text{applico} \frac{1}{1-q}
    \]

   \begin{align*}
        &\sum_{n=1}^{\infty} \left( -\frac{1}{2} \right)^{n-1}  = \frac{1}{1 - \left(-\frac{1}{2}\right)} =  \frac{3}{2} \\
        &\sum_{n=1}^{\infty} \left( \frac{1}{3} \right)^{n-1}  = \frac{1}{1 - \frac{1}{3}} = \frac{2}{3} \\
        &\sum_{n=1}^{\infty} \left[\left( -\frac{1}{2}\right)^{n-1} + \left (\frac{1}{3}\right )^{n-1} \right] = \frac{2}{3}+\frac{3}{2} = \frac{13}{6} \\
    \end{align*}

   \section*{Serie prodotto}

\begin{align*}
    &\text{Siano } \sum_{n=1}^{\infty} a_n \quad \text{con termini } a_1 + a_2 + a_3 + a_4 + a_5 + \cdots + a_n + \cdots \\
    &\text{e una costante } \lambda \in \mathbb{R}. \\
    &\text{La serie } \sum_{n=1}^{\infty} \left(\lambda a_n\right) \quad \textbf{è una Serie Prodotto, data da:} \\
    &\lambda a_1 + \lambda a_2 + \lambda a_3 + \cdots + \lambda a_n + \cdots \\
\end{align*}

Definiamo la somma parziale \( n \)-esima di ciascuna serie come segue:

\begin{align*}
    A_n &= a_1 + a_2 + \cdots + a_n, \quad \text{somma parziale della serie } \sum_{n=1}^{\infty} a_n, \\
    P_n &= \lambda a_1 + \lambda a_2 + \cdots + \lambda a_n, \quad \text{somma parziale della serie } \sum_{n=1}^{\infty} \lambda a_n.
\end{align*}

Osserviamo che:
\[
    P_n = \lambda \left(a_1 + a_2 + \cdots + a_n\right) = \lambda A_n, \quad \forall n \in \mathbb{N}.
\]
\begin{align*}
    \text{Quindi, se la serie } \sum_{n=1}^{\infty} a_n \text{ converge a } S, \text{ allora la serie prodotto } \\ \sum_{n=1}^{\infty} \lambda a_n \text{ converge a } \lambda S. \\
\end{align*}

Si hanno due casi:
\begin{enumerate}
    \item Se $\lambda = 0$, allora $\sum_{n=1}^{+\infty} \left( \lambda a_n \right) = 0$.
    \item Se $\lambda \neq 0$
\end{enumerate}

Se $\lambda \neq 0$, si hanno i seguenti casi:

$
\begin{cases}
    \sum_{n=1}^{\infty} a_n = A \in \mathbb{R} & \Rightarrow \sum_{n=1}^{\infty} \left( \lambda a_n \right) = \lambda A \Leftrightarrow \sum_{n=1}^{\infty} \left( \lambda a_n \right) = \lambda \sum_{n=1}^{\infty} a_n \\
    \sum_{n=1}^{\infty} a_n = \pm\infty & \Rightarrow \sum_{n=1}^{\infty} \left( \lambda a_n \right) = 
    \begin{cases}
        +\infty & \text{se } \lambda > 0 \\
        -\infty & \text{se } \lambda < 0
    \end{cases} 
    \\
\end{cases} $

\subsection*{Esercizi Serie Prodotto}
\begin{align*}
    \sum_{n=1}^{\infty} 3 \left(-\frac{1}{4}\right)^{n-1} &= \frac{3}{1 - \left( -\frac{1}{4}\right)} = \frac{3}{1 + \frac{1}{4}} = \frac{3}{\frac{5}{4}} = \frac{3 \cdot 4}{5} = \frac{12}{5}
\end{align*}

\section* {Serie resto}
\begin{align*}
    &\text{Sia data una serie numerica:} \\
    &a_1 + a_2 + \dots + a_n + \dots = \sum_{n=1}^{+\infty} a_n, \\
    &\text{e sia } \nu \in \mathbb{N} \text{ un indice naturale.} \\
    \\
    &\text{Definiamo la \textbf{serie resto di posto} } \nu \text{ come la serie formata dai termini successivi a } a_\nu, \text{ ovvero:} \\
    &a_{\nu+1} + a_{\nu+2} + \dots + a_{\nu+k} + \dots = \sum_{n=1}^{+\infty} a_{\nu+n}. \\
    \\
    &\text{In notazione alternativa, possiamo anche scriverla come:} \\
    &\sum_{n=\nu+1}^{+\infty} a_n. \\
    \\
    &\text{Questa serie rappresenta la \emph{parte restante} della serie originale, partendo dall'indice } \nu+1 \text{ in poi.} \\
    &\text{Studiare la serie resto è utile per analizzare il comportamento della convergenza della serie iniziale.} \\
\end{align*}

\subsection*{Esempi di serie resto}

\begin{align*}
    &\text{Se scegliamo } \nu = 3, \text{ la serie resto di posto 3 è data da:} \\
    &a_4 + a_5 + a_6 + \dots = \sum_{h=4}^{+\infty} a_h = \sum_{n=1}^{+\infty} a_{3+n}.
\\
\end{align*}
\subsection*{Teorema sul resto}

\begin{align*}
    &\text{In particolare, nel caso di convergenza, possiamo scrivere la somma di una serie come:} \\
    &\sum_{n=1}^{+\infty} a_n = \underbrace{\left(a_1 + \dots + a_\nu\right)}_{\text{somma dei primi $\nu$ termini}} + \underbrace{\sum_{n=\nu+1}^{+\infty} a_n}_{\text{somma della serie resto di posto $\nu$}}. \\
    & \sum_{n=\nu+1}^{+\infty} a_n = \sum_{n=1}^{\infty} a_n - \left( a_1+\dots+ a_\nu) \right    \\
\end{align*} 

\subsection*{Esempio di applicazione del teorema del resto}

\begin{align*}
    &\sum_{n=1}^{\infty} \left( \frac{1}{2} \right)^{n+2} \\
    &= \left( \frac{1}{2} \right)^3 + \left( \frac{1}{2} \right)^4 + \cdots + \left( \frac{1}{2} \right)^{n+2} \\
    &\text{è la serie resto di posto } \nu = 2 \text{ della serie geometrica di ragione } \frac{1}{2}.
\end{align*}

\begin{align*}
    \sum_{n=1}^{+\infty} \left( \frac{1}{2} \right)^{n+2} &= \sum_{n=1}^{+\infty} \left( \frac{1}{2} \right)^{n-1} - \left( 1 + \frac{1}{2} + \frac{1}{4} \right) \\
    &= \frac{1}{1 - \left( \frac{1}{2} \right)} - \left( 1 + \frac{1}{2} + \frac{1}{4} \right) \\
    &= 2 - 1 - \frac{1}{2} - \frac{1}{4} \\
    &= \frac{4 - 2 - 1 - 1}{4} = \frac{1}{4}.
\end{align*}

\section*{Serie a termini non negativi}
\begin{align*}
    &\text{Sia} \\
    &\sum_{n=1}^{\infty} a_n \\
    &\text{una serie numerica con $a_n \geq 0 \; (a_n > 0)$} \quad \forall n \in \mathbb{N}, \\
    &\text{detta a termini non negativi (o positivi).}
\end{align*}

\subsection*{Regolarità delle serie a termini non negativi (Teorema)}
\textbf{Ogni serie a termini non negativi è regolare, cioè converge oppure diverge positivamente.}

\subsection*{Dimostrazione}
Si dimostra che la successione delle somme parziali è crescente:
\begin{align*}
    &\forall n \in \mathbb{N} \quad \text{si ha:} \\
    &A_n = a_1 + a_2 + \cdots + a_n, \\
    &A_{n+1} = a_1 + a_2 + \cdots + a_n + a_{n+1} = A_n + \underbrace{a_{n+1}}_{\geq 0} \geq A_n.
\end{align*}
Da ciò segue che:
\begin{align*}
    &\{A_n\} \quad \text{è una successione crescente.} \\
    &\text{Per il teorema sul limite delle successioni monotone,} \\
    &\lim_{n \to \infty} A_n = \sup A_n \begin{cases}
        \in \mathbb{R}, & \text{se la serie converge}, \\
        +\infty, & \text{se la serie diverge positivamente.}
    \end{cases}
\end{align*}
Per le serie a termini non negativi convergenti, ogni somma parziale è $\leq$ della somma della serie:
\begin{align*}
    &\text{In simboli:} \\
    &\text{Se } \sum_{n=1}^{+\infty} a_n = A \in \mathbb{R} \quad \text{e} \quad a_n \geq 0 \; \forall n \in \mathbb{N}, \\
    &A = \sup \{A_n\} \to A_n \leq A \quad \forall n \in \mathbb{N}
\end{align*}

\subsection*{Esempi di serie a termini non negativi}
\begin{align*}
    &\sum_{n=1}^{+\infty} \frac{n}{n+1} \quad \text{è una serie a termini positivi e può convergere o divergere positivamente.} \\
    &\lim_{n \to +\infty} \frac{n}{n+1} = 1 \neq 0 \quad \Rightarrow \quad \sum_{n=1}^{+\infty} \frac{n}{n+1} \quad \textbf{non converge.} \\
    &\text{Da ciò segue che la serie diverge positivamente.}
\end{align*}

\begin{align*}
    &\sum_{n=1}^{+\infty} a_n \geq 0 \quad \forall n \in \mathbb{N}, \\
    &\lim_{n \to +\infty} a_n \neq 0.
\end{align*}

\section*{Notazione}
\begin{align*}
    &\text{Siano:} \\
    &\sum_{n=1}^{+\infty} a_n \quad \text{e} \quad \sum_{n=1}^{+\infty} b_n \quad \text{due serie numeriche tali che:} \quad a_n \leq b_n \quad \forall n \in \mathbb{N}. \\
    &\text{Si ha che:}
\end{align*}

\begin{enumerate}
    \item $\sum_{n=1}^{+\infty} a_n$ \text{è maggiorata da} $\sum_{n=1}^{+\infty} b_n$
    \item $\sum_{n=1}^{+\infty} a_n$ \text{è minorata da} $\sum_{n=1}^{+\infty} b_n$
\end{enumerate}


\subsection*{Dimostrazione del Teorema del Confronto sulle Serie Numeriche}
Siano, per ogni \( n \in \mathbb{N} \):
\[
A_n = a_1 + a_2 + \dots + a_n, \quad B_n = b_1 + b_2 + \dots + b_n.
\]

\noindent
Dimostriamo che, se \( a_n, b_n \geq 0 \ \forall n \) e \( a_n \leq b_n \ \forall n \), allora \( \sum_{n=1}^{+\infty} b_n \) converge implica che \( \sum_{n=1}^{+\infty} a_n \) converge.

\begin{proof}
Sia:
\[
a_n \leq b_n \quad \forall n \in \mathbb{N}.
\]
Da ciò segue che le successioni delle somme parziali \( \{A_n\} \) e \( \{B_n\} \) sono crescenti e:
\[
A_n \leq B_n \quad \forall n \in \mathbb{N}. \tag{*}
\]

\noindent
Supponiamo che \( \sum_{n=1}^{+\infty} b_n \) converga. Questo implica che la successione \( \{B_n\} \) è limitata. Poniamo:
\[
B = \sup \{B_n\} = \sum_{n=1}^{+\infty} b_n.
\]

\noindent
Poiché \( B_n \leq B \ \forall n \in \mathbb{N} \), dalla \((*)\) segue che:
\[
A_n \leq B_n \leq B \quad \forall n \in \mathbb{N}. \tag{**}
\]

\noindent
Da \((**)\) segue che \( B \) è un maggiorante di \( \{A_n\} \). Quindi, \( \{A_n\} \) è una successione crescente e limitata superiormente, e di conseguenza converge. Ponendo:
\[
A = \lim_{n \to +\infty} A_n = \sup \{A_n\},
\]
abbiamo:
\[
\sup \{A_n\} \leq B \quad \Rightarrow \quad A = \sum_{n=1}^{+\infty} a_n \leq \sum_{n=1}^{+\infty} b_n.
\]

\noindent
Da \((**)\) segue che:
\[
a_n \geq b_n \quad \forall n \in \mathbb{N} \quad \Rightarrow \quad A_n \geq B_n \quad \forall n \in \mathbb{N}.
\]

\noindent
Se \( \sum_{n=1}^{\infty} b_n \) diverge positivamente, allora:
\[
\lim_{n \to +\infty} B_n = +\infty.
\]

\noindent
Da ciò segue che:
\[
\lim_{n \to +\infty} A_n = +\infty \quad \Longleftrightarrow \quad \sum_{n=1}^{\infty} a_n \text{ diverge positivamente.}
\]

\noindent
Questo conclude la dimostrazione.
\end{proof}

\section*{Criterio del confronto asintotico}
\begin{align*}
    &\text{Siano:} \\
    &\sum_{n=1}^{\infty} a_n \quad \text{e} \quad \sum_{n=1}^{\infty} b_n, \quad \text{con} \quad a_n \geq 0, \quad b_n > 0 \quad \forall n \in \mathbb{N}.
\end{align*}

\begin{enumerate}
    \item[\( \text{i)} \)] Se 
    \(\lim_{n \to +\infty} \frac{a_n}{b_n} = l \in \mathbb{R}^+\), allora 
    \(\sum_{n=1}^{\infty} a_n \quad \text{e} \quad \sum_{n=1}^{\infty} b_n\) 
    hanno lo stesso carattere (ovvero entrambe convergono oppure entrambe divergono).

    \item[\( \text{ii)} \)] Se 
    \(\lim_{n \to +\infty} \frac{a_n}{b_n} = 0\) e 
    \(\sum_{n=1}^{\infty} b_n\) converge, allora 
    \(\sum_{n=1}^{\infty} a_n\) converge.

    \item[\( \text{iii)} \)] Se 
    \(\lim_{n \to +\infty} \frac{a_n}{b_n} = +\infty\) e 
    \(\sum_{n=1}^{\infty} b_n\) diverge positivamente, allora 
    \(\sum_{n=1}^{\infty} a_n\) diverge positivamente.
\end{enumerate}
 \subsection*{Dimostrazione del criterio del confronto asintotico}
Basta osservare che 
\[
\frac{a_n}{\frac{1}{n^x}} \to l
\]
ed applicare il Criterio del confronto asintotico.

Il criterio appena esposto è anche detto \textbf{Criterio dell'ordine di infinitesimo} perché, se \(\{a_n\}\) è infinitesima, dal fatto che 
\[
n^x a_n \to l > 0
\]
segue che 
\[
a_n \sim \frac{l}{n^x},
\]
ossia che \(\{a_n\}\) è infinitesimo di ordine \(x\) rispetto all'infinitesimo campione \(\left\{\frac{1}{n}\right\}\).

Utilizzando i criteri sulle serie a termini non negativi, siamo in grado di introdurre una \textbf{classe di serie} 
 




\end{document}
