
\documentclass{article}
\usepackage{amsmath}
\usepackage{amsfonts}
\usepackage{amssymb}
\usepackage{geometry}
\geometry{a4paper, margin=1in}

\title{Verifica di Continuità, Derivate Parziali e Differenziabilità di una Funzione di Due Variabili}
\author{}
\date{}

\begin{document}

\maketitle

\section*{Introduzione}
In questo documento, analizziamo la funzione \( f(x, y) = x^2 + y^2 \ln(x^2 + y^2) \) per verificarne la continuità, calcolare le derivate parziali e determinarne la differenziabilità nel punto \( (0, 0) \).

\section{Continuità}
Per verificare la continuità della funzione \( f(x, y) \) nel punto \( (0, 0) \), calcoliamo il limite di \( f(x, y) \) quando \( (x, y) \to (0, 0) \) lungo vari percorsi, e vediamo se tutti i limiti concordano:
\begin{equation*}
f(x, y) = x^2 + y^2 \ln(x^2 + y^2).
\end{equation*}

Consideriamo il percorso lungo l'asse \(x\), cioè \( y = 0 \):
\begin{equation*}
f(x, 0) = x^2.
\end{equation*}
Quando \( x \to 0 \), si ha \( f(x, 0) \to 0 \).

Ora consideriamo il percorso lungo l'asse \(y\), cioè \( x = 0 \):
\begin{equation*}
f(0, y) = y^2 \ln(y^2).
\end{equation*}
Per \( y \to 0 \), il termine \( y^2 \ln(y^2) \to 0 \) poiché \( y^2 \) tende a zero più velocemente di \( \ln(y^2) \) tende a infinito. 

Quindi, in entrambi i casi si ha:
\begin{equation*}
\lim_{(x, y) \to (0, 0)} f(x, y) = 0 = f(0, 0).
\end{equation*}
Pertanto, la funzione \( f(x, y) \) è continua in \( (0, 0) \).

\section{Derivate Parziali}
Calcoliamo ora le derivate parziali di \( f(x, y) \) rispetto a \( x \) e \( y \).

\subsection*{Derivata Parziale rispetto a \( x \)}
La derivata parziale di \( f \) rispetto a \( x \) è data da:
\begin{equation*}
f_x(x, y) = \frac{\partial}{\partial x} \left( x^2 + y^2 \ln(x^2 + y^2) \right).
\end{equation*}
Applicando la regola del prodotto e la derivata del logaritmo otteniamo:
\begin{equation*}
f_x(x, y) = 2x + \frac{2x y^2}{x^2 + y^2} = 2x \left(1 + \ln(x^2 + y^2)\right).
\end{equation*}

\subsection*{Derivata Parziale rispetto a \( y \)}
La derivata parziale di \( f \) rispetto a \( y \) è data da:
\begin{equation*}
f_y(x, y) = \frac{\partial}{\partial y} \left( x^2 + y^2 \ln(x^2 + y^2) \right).
\end{equation*}
Calcolando otteniamo:
\begin{equation*}
f_y(x, y) = 2y + \frac{2y x^2}{x^2 + y^2} = 2y \left(1 + \ln(x^2 + y^2)\right).
\end{equation*}

\section{Continuità delle Derivate Parziali}
Per verificare la differenziabilità, analizziamo il comportamento delle derivate parziali \( f_x(x, y) \) e \( f_y(x, y) \) vicino a \( (0, 0) \).

Consideriamo il limite delle derivate parziali \( f_x(x, y) \) e \( f_y(x, y) \) quando \( (x, y) \to (0, 0) \).

\begin{equation*}
f_x(x, y) = 2x \left(1 + \ln(x^2 + y^2)\right) \to 0 \quad \text{quando} \quad (x, y) \to (0, 0),
\end{equation*}
\begin{equation*}
f_y(x, y) = 2y \left(1 + \ln(x^2 + y^2)\right) \to 0 \quad \text{quando} \quad (x, y) \to (0, 0).
\end{equation*}
Pertanto, \( f_x \) e \( f_y \) sono continue in \( (0, 0) \).

\section{Differenziabilità}
Poiché \( f \) è continua in \( (0, 0) \) e le derivate parziali \( f_x \) e \( f_y \) sono continue in \( (0, 0) \), possiamo concludere che \( f \) è differenziabile in \( (0, 0) \). Il differenziale totale di \( f \) in \( (0, 0) \) è dato da:
\begin{equation*}
df = f_x(0, 0) \, dx + f_y(0, 0) \, dy = 0 \cdot dx + 0 \cdot dy = 0.
\end{equation*}
Ciò significa che, vicino al punto \( (0, 0) \), la funzione può essere approssimata da una funzione costante (in prima approssimazione).

\end{document}
